\documentclass{beamer}
\usepackage{lmodern}
\usepackage[utf8]{inputenc}
\usepackage{hyperref}
\usepackage{minted}
\usepackage{color}

\setbeamertemplate{navigation symbols}{}
\mode<presentation>{\usetheme{Madrid}}

\title{Named Entity Classification}   
\author{Madita Huvar, Sanaz Safdel, Phillip R.-P.} 
\date{\today} 

\begin{document}



\begin{frame}
\titlepage
\end{frame} 

\begin{frame}
\frametitle{Inhaltsverzeichnis}
\tableofcontents
\end{frame} 


\section{Einführung}
\section{Daten \& Tools}
	\subsection{Tools}
	\begin{frame}
			\frametitle{Tools}
			\begin{itemize}
				\item Python 3.4+
				\item Scikit Learn als Klassifizierer
				\item liac-arff
				\item Weka zur Korpusanalyse
			\end{itemize}
	\end{frame}
	\subsection{Korpus}
	\begin{frame}
			\frametitle{Korpus}
			
	\end{frame}
	\subsection{Korpusklassen}
	\begin{frame}
			\frametitle{Korpusklassen}
			
	\end{frame}


\section{Klassifizierer}
	\subsection{Features für den Baseline-Klassifizierer}
	\begin{frame}
		\frametitle{Features für den Baseline-Klassifizierer}
		
	\end{frame}
	\subsection{Erweitertes Featureset}
	\begin{frame}
		\frametitle{Erweitertes Featureset}
		
	\end{frame}
	\subsection{Klassifizierertyp}
	\begin{frame}
		\frametitle{Klassifizierertyp}
		
	\end{frame}
	\subsection{Probleme}
	\begin{frame}
		\frametitle{Probleme}
		
	\end{frame}
	\subsection{Erfahrungen mit den Korpusklassen}
	\begin{frame}
		\frametitle{Erfahrungen mit den Korpusklassen}
		
	\end{frame}
\section{Evaluation}
	\begin{frame}
		\frametitle{Evaluation}
		
	\end{frame}
\section{Ausblick}
	\begin{frame}
		\frametitle{Ausblick}
		
	\end{frame}
\section{Referenzen}
	\begin{frame}
		\frametitle{Referenzen}
		
		\end{frame}

\end{document}