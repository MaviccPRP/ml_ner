\documentclass{beamer}
\usepackage{lmodern}
\usepackage[utf8]{inputenc}
\usepackage{hyperref}
\usepackage{minted}
\usepackage{color}
\usepackage{booktabs}
\usepackage{tabularx}

\setbeamertemplate{navigation symbols}{}
\mode<presentation>{\usetheme{Madrid}}

\title{Named Entity Classification}   
\author{Madita Huvar, Sanaz Safdel, Phillip R.-P.} 
\date{\today} 

\begin{document}



\begin{frame}
\titlepage
\end{frame} 

\begin{frame}
\frametitle{Inhaltsverzeichnis}
\tableofcontents
\end{frame} 


\section{Einführung}
\section{Daten \& Tools}
	\subsection{Tools}
	\begin{frame}
			\frametitle{Tools}
			\begin{itemize}
				\item Python 3.4+
				\item Scikit Learn als Klassifizierer
				\item liac-arff
				\item Weka zur Korpusanalyse
			\end{itemize}
	\end{frame}
	\subsection{Korpus}
	\begin{frame}
			\frametitle{Korpus}
			Für die Named Entity Klassifikation nutzen wir das OntoNotes Korpus 2012.\\
			
			Dabei nutzen die englischen Nachrichtentexte des The Wall Street Journal.
			Für die Entwicklungsphase nutzten wir das im OntoNotes Korpus bereits vorgefertigte Developmenttest.\\
			
			Für die Klassifikation der Named Entities werden die bereits vorgefertigten Trainings- und Testdatensets genutzt.\\
			 \begin{table}
			 	\caption{Anzahl an atomaren Named Entities}
			 	\begin{tabular}{ccc}
			 		\toprule
					Developmentset & Trainingset & Testset\\
			 		\midrule
					3325 & 23686 & 2996\\
			 		\bottomrule
			 	\end{tabular}
			 	\label{tab:datasets}
			 \end{table}
	\end{frame}
		\begin{frame}
			\frametitle{Korpusreader}
			-
		\end{frame}
	
	\subsection{Korpusklassen}
		\begin{frame}
			\frametitle{Korpusklassenbalancierung}
			-
		\end{frame}
	\begin{frame}
		\frametitle{Beschreibung Korpusklassen}
		\begin{table}
			\caption{Balancierte Klassen}
			\begin{tabularx}{\textwidth}{lX}
				\toprule
				Klassen  & Beschreibung \\
				\midrule
				PERSON 	& People, including fictional \\
				NORP\_GPE &	Nationalities or religious or political groups
				Countries, cities, states\\
				ORGANIZATION &	Companies, agencies, institutions, etc.\\
				DATE &	Absolute or relative dates or periods\\
				PERCENT\_MONEY\_CARDINAL &	Percentage (including “\%”)
				Monetary values, including unit
				Numerals that do not fall under another type \\
				\bottomrule
			\end{tabularx}
			\label{tab:datasets}
		\end{table}
	\end{frame}
	
	\begin{frame}
			\frametitle{Verteilung Korpusklassen}
			 \begin{table}
			 	\caption{Verteilung der Klassen}
			 	\begin{tabular}{lccc}
			 		\toprule
			 		Klassen  & Developmentset & Trainingset & Testset \\
			 		\midrule
			 		ORG  & 930 & 5857 & 859 \\
			 		PERSON & 486 & 3759 & 413 \\
			 		GPE\_NORP & 732 & 5134 & 588 \\
			 		PERCENT\_CARDINAL\_MONEY & 564 & 4672 & 529 \\
			 		DATE & 613 & 4254 & 601 \\
			 		\bottomrule
			 	\end{tabular}
			 	\label{tab:datasets}
			 \end{table}
	\end{frame}


\section{Klassifizierer}
	\subsection{Features für den Baseline-Klassifizierer}
	\begin{frame}
		\frametitle{Features für den Baseline-Klassifizierer}
					 \begin{table}
					 	\caption{Features für den Baseline-Klassifizierer}
					 	\begin{tabularx}{\textwidth}{llX}
					 		\toprule
							Feature & Wert & Beschreibung\\
					 		\midrule
					 		Unigram & numerisch & Vorkommen der Unigramme, die mindestens fünfmal im Trainingscorpus vorkommen \\
					 		\bottomrule
					 	\end{tabularx}
					 	\label{tab:datasets}
					 \end{table}
	\end{frame}
	\subsection{Erweitertes Featureset}
	\begin{frame}
		\frametitle{Erweitertes Featureset}
		
	\end{frame}
	\subsection{Klassifizierertyp}
	\begin{frame}
		\frametitle{Klassifizierertyp}
		
	\end{frame}
	\subsection{Probleme}
	\begin{frame}
		\frametitle{Probleme}
		
	\end{frame}
	\subsection{Erfahrungen mit den Korpusklassen}
	\begin{frame}
		\frametitle{Erfahrungen mit den Korpusklassen}
		
	\end{frame}
\section{Evaluation}
	\begin{frame}
		\frametitle{Evaluation}
		
	\end{frame}
\section{Ausblick}
	\begin{frame}
		\frametitle{Ausblick}
		
	\end{frame}
\section{Referenzen}
	\begin{frame}
		\frametitle{Referenzen}
		
		\end{frame}

\end{document}