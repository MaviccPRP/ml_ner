\documentclass{beamer}
\usepackage{lmodern}
\usepackage[utf8]{inputenc}
\usepackage{hyperref}
\usepackage{minted}
\usepackage{color}
\usepackage{booktabs}
\usepackage{tabularx}
\usepackage{array}

\setbeamertemplate{navigation symbols}{}
\mode<presentation>{\usetheme{Madrid}}

\title{Named Entity Classification}   
\author{Madita Huvar, Sanaz Safdel, Phillip R.-P.} 
\date{\today} 

\begin{document}



\begin{frame}
\titlepage
\end{frame} 

\begin{frame}
\frametitle{Inhaltsverzeichnis}
\tableofcontents
\end{frame} 


\section{Einführung}
\section{Daten \& Tools}
	\subsection{Tools}
	\begin{frame}
			\frametitle{Tools}
			\begin{itemize}
				\item Python 3.4+
				\item Scikit Learn als Klassifizierer
				\item liac-arff
				\item Weka zur Korpusanalyse
			\end{itemize}
	\end{frame}
	\subsection{Korpus}
	\begin{frame}
			\frametitle{Korpus}
			Für die Named Entity Klassifikation nutzen wir das OntoNotes Korpus 2012.\\
			
			Dabei nutzen die englischen Nachrichtentexte des The Wall Street Journal.
			Für die Entwicklungsphase nutzten wir das im OntoNotes Korpus bereits vorgefertigte Developmenttest.\\
			
			Für die Klassifikation der Named Entities werden die bereits vorgefertigten Trainings- und Testdatensets genutzt.\\
			 \begin{table}
			 	\caption{Anzahl an atomaren Named Entities}
			 	\begin{tabular}{ccc}
			 		\toprule
					Developmentset & Trainingset & Testset\\
			 		\midrule
					3325 & 23686 & 2996\\
			 		\bottomrule
			 	\end{tabular}
			 	\label{tab:datasets}
			 \end{table}
	\end{frame}
		\begin{frame}
			\frametitle{Korpusreader}
			Für die Extraktion der Named Entities wurde ein Korpusreader erstellt.\\
			Der Reader extrahiert alle Named Entities, inklusive POS-tags der einzelnen Tokenm, Phrasenart, Kontextwörtern (ne-1, ne+1), und ordnet sie ihren Klassen zu.\\
			
			Beispiel:\\
			\{'PERSON':[['Peter', 'NNP'],['Mokaba', 'NNP'],'NP', ('Says', ',')]\}
		\end{frame}
	
	\subsection{Korpusklassen}
		\begin{frame}
			\frametitle{Korpusklassenbalancierung}
				\begin{table}
				\caption{Klassen im OntoNotes Korpus}
				\begin{tabularx}{\textwidth}{Xc}
					\toprule
					Klassen  & Trainingset \\
					\midrule
					ORG  & 5788 \\
					PERSON & 3756 \\
					GPE & 3601 \\
					NORP & 1484 \\
					PERCENT & 1061  \\
					CARDINAL & 1852 \\
					MONEY & 1509  \\
					DATE & 4080  \\
					FAC, LOC, PRODUCT, EVENT, WORK\_OF\_ART, LAW, LANGUAGE, TIME, QUANTITY, ORDINAL & $<$ 1800 \\
					\bottomrule
				\end{tabularx}
				\label{tab:datasets}
			\end{table}
		\end{frame}
	\begin{frame}
		\frametitle{Beschreibung Korpusklassen}
		\begin{table}
			\caption{Balancierte Klassen}
			\begin{tabularx}{\textwidth}{lX}
				\toprule
				Klassen  & Beschreibung \\
				\midrule
				PERSON 	& People, including fictional \\
				NORP\_GPE &	Nationalities or religious or political groups
				Countries, cities, states\\
				ORGANIZATION &	Companies, agencies, institutions, etc.\\
				DATE &	Absolute or relative dates or periods\\
				PERCENT\_MONEY\_CARDINAL &	Percentage (including “\%”)
				Monetary values, including unit
				Numerals that do not fall under another type \\
				\bottomrule
			\end{tabularx}
			\label{tab:datasets}
		\end{table}
	\end{frame}
	
	\begin{frame}
			\frametitle{Verteilung Korpusklassen}
			 \begin{table}
			 	\caption{Verteilung der Klassen}
			 	\begin{tabular}{lccc}
			 		\toprule
			 		Klassen  & Developmentset & Trainingset & Testset \\
			 		\midrule
			 		ORG  & 930 & 5857 & 859 \\
			 		PERSON & 486 & 3759 & 413 \\
			 		GPE\_NORP & 732 & 5134 & 588 \\
			 		PERCENT\_CARDINAL\_MONEY & 564 & 4672 & 529 \\
			 		DATE & 613 & 4254 & 601 \\
			 		\bottomrule
			 	\end{tabular}
			 	\label{tab:datasets}
			 \end{table}
	\end{frame}


\section{Klassifizierer}
	\subsection{Features für den Baseline-Klassifizierer}
	\begin{frame}
		\frametitle{Features für den Baseline-Klassifizierer}
					 \begin{table}
					 	\caption{Features für den Baseline-Klassifizierer}
					 	\begin{tabularx}{\textwidth}{llX}
					 		\toprule
							Feature & Wert & Beschreibung\\
					 		\midrule
					 		Unigram & numerisch & Vorkommen der Unigramme, die mindestens fünfmal im Trainingscorpus vorkommen \\
					 		\bottomrule
					 	\end{tabularx}
					 	\label{tab:baselinef}
					 \end{table}
	\end{frame}
	\subsection{Erweitertes Featureset}
	\begin{frame}
		\frametitle{Erweitertes Featureset I}
		Anzahl der Features: 1716
 			\begin{table}
 				\caption{Features für den Baseline-Klassifizierer I}
 				\begin{tabularx}{\textwidth}{llX}
 					\toprule
 					Feature & Wert & Beschreibung\\
 					\midrule
 					Unigram & numerisch & Vorkommen der Unigramme (lemmatisiert), die mindestens fünfmal im Trainingscorpus vorkommen \\
 					POS & numerisch & Häufigkeit von 36 POS-Tags aus der Penn Treebank\\
 					isAllCaps & boolean & Wörter nur in Großschreibung\\
 					Context & numerisch & Vorkommen der Kontexttokens, die mindestens fünfmal im Trainingskorpus vorkommen. Das Kontextfenster beinhaltet das Vorgänger- und Nachfolgetoken der NE.\\
 					containsDigit & boolean & Vorkommen von Nummern.\\
 					\bottomrule
 				\end{tabularx}
 				\label{tab:allf1}
 			\end{table}
 	\end{frame}
 		\begin{frame}
 			\frametitle{Erweitertes Featureset II}
 					\begin{table}
 						\caption{Features für den Baseline-Klassifizierer II}
 						\begin{tabularx}{\textwidth}{llX}
 							\toprule
 							Feature & Wert & Beschreibung\\
 							\midrule
 							isInWiki & boolean & Vorkommen der NE in der Wikipedia. \\
 							isTitle & boolean & Prüft, ob Titelbezeichnungen (z.B. Mr. MA) vorkommen.\\
 							isNP & boolean & Ist NE eine Nominalphrase \\
 							isName & boolean & Prüft, ob Vornamen vorkommen..\\
 							containsDash & boolean & Vorkommen von Viertelgeviertstrichen\\
 							\bottomrule
 						\end{tabularx}
 						\label{tab:allf2}
 					\end{table}
 		\end{frame}
	\subsection{Klassifizierertyp}
	\begin{frame}
		\frametitle{Klassifizierertyp}
		Zur Klassifizierung der NE werden zwei verschiedene Klassifizierer genutzt.\\
		
		\begin{itemize}
			\item SVM (sklearn.svm.LinearSVC)\\
				\em{Parameter: Dual: True, loss: squared hinge, class weight: balanced}
			\item DecisionTree (sklearn.tree.DecisionTreeClassifier)\\
			\em{Parameter: class weight: balanced}
		\end{itemize}
	\end{frame}
	\subsection{Probleme}
	\begin{frame}
		\frametitle{Probleme}
		
	\end{frame}
	\subsection{Erfahrungen mit den Korpusklassen}
	\begin{frame}
		\frametitle{Erfahrungen mit den Korpusklassen}
		
	\end{frame}
\section{Evaluation}
	\begin{frame}
		\frametitle{Evaluation}
		
	\end{frame}
\section{Ausblick}
	\begin{frame}
		\frametitle{Ausblick}
		
	\end{frame}
\section{Referenzen}
	\begin{frame}
		\frametitle{Referenzen}
		\begin{itemize}
			\item Cho, Han-Cheol; Okazaki, Naoaki; Miwa, Makoto; Tsujii, Jun’ichi (2013): Named entity recognition with multiple segment representations. In: Information Processing \& Management 49\\
			\item Derczynski, Leon; Maynard, Diana; Rizzo, Giuseppe; van Erp, Marieke; Gorrell, Genevieve; Troncy, Raphaël et al. (2015): Analysis of named entity recognition and linking for tweets. In: Information Processing \& Management 51 (2), S. 32–49.\\
			\item Konkol, Michal; Brychcín, Tomáš; Konopík, Miloslav (2015): Latent semantics in Named Entity Recognition. In: Expert Systems with Applications 42 (7), S. 3470–3479.\\
			\item Agerri, Rodrigo; Rigau, German (2016): Robust multilingual Named Entity Recognition with shallow semi-supervised features. In: Artificial Intelligence 238, S. 63–82.\\
			\item Erik F. Tjong Kim Sang and Fien De Meulder (2003): Language-Independent Named Entity Recognition.\\
		\end{itemize}
	\end{frame}
	\begin{frame}
		\begin{itemize}
			\item Marrero, Mónica; Urbano, Julián; Sánchez-Cuadrado, Sonia; Morato, Jorge; Gómez-Berbís, Juan Miguel (2013): Named Entity Recognition. Fallacies, challenges and opportunities. In: Computer Standards \& Interfaces 35 (5), S. 482–489.\\
			\item Mayfield, James; McNamee, Paul; Piatko, Christine (2003): Named entity recognition using hundreds of thousands of features. In: Walter Daelemans und Miles Osborne (Hg.): Proceedings of the seventh conference on Natural language learning at HLT-NAACL 2003 -. the seventh conference. Edmonton, Canada. Morristown, NJ, USA: Association for Computational Linguistics, S. 184–187.\\
			\item Mónica Marrero, Sonia Sánchez-Cuadrado (2009): Evaluation of Named Entity Extraction Systems.\\
			\item Weischedel, Ralph M. (2013): OntoNotes release 5.0. [Philadelphia, Pa.]: Linguistic Data Consortium.\\
		\end{itemize}
	\end{frame}

\end{document}
